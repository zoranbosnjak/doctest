%% Generated by Sphinx.
\def\sphinxdocclass{report}
\documentclass[a4paper,11pt,english]{sphinxmanual}
\ifdefined\pdfpxdimen
   \let\sphinxpxdimen\pdfpxdimen\else\newdimen\sphinxpxdimen
\fi \sphinxpxdimen=.75bp\relax
\ifdefined\pdfimageresolution
    \pdfimageresolution= \numexpr \dimexpr1in\relax/\sphinxpxdimen\relax
\fi
%% let collapsible pdf bookmarks panel have high depth per default
\PassOptionsToPackage{bookmarksdepth=5}{hyperref}
%% turn off hyperref patch of \index as sphinx.xdy xindy module takes care of
%% suitable \hyperpage mark-up, working around hyperref-xindy incompatibility
\PassOptionsToPackage{hyperindex=false}{hyperref}
%% memoir class requires extra handling
\makeatletter\@ifclassloaded{memoir}
{\ifdefined\memhyperindexfalse\memhyperindexfalse\fi}{}\makeatother

\PassOptionsToPackage{booktabs}{sphinx}
\PassOptionsToPackage{colorrows}{sphinx}

\PassOptionsToPackage{warn}{textcomp}

\catcode`^^^^00a0\active\protected\def^^^^00a0{\leavevmode\nobreak\ }
\usepackage{cmap}
\usepackage{fontspec}
\defaultfontfeatures[\rmfamily,\sffamily,\ttfamily]{}
\usepackage{amsmath,amssymb,amstext}
\usepackage{polyglossia}
\setmainlanguage{english}


\setmainfont{DejaVu Serif}
\setsansfont{DejaVu Sans}
\setmonofont{DejaVu Sans Mono}




\usepackage[Bjarne]{fncychap}
\usepackage{sphinx}
\sphinxsetup{VerbatimColor={rgb}{0.93, 1.00, 0.80}}
\fvset{fontsize=\small}
\usepackage{geometry}


% Include hyperref last.
\usepackage{hyperref}
% Fix anchor placement for figures with captions.
\usepackage{hypcap}% it must be loaded after hyperref.
% Set up styles of URL: it should be placed after hyperref.
\urlstyle{same}

\addto\captionsenglish{\renewcommand{\contentsname}{Contents:}}

\usepackage{sphinxmessages}
\setcounter{tocdepth}{5}
\setcounter{secnumdepth}{5}
% \usepackage{colortbl}     % for tables (not needed with current Sphinx)
\usepackage{paralist}       % for compact itemize
\usepackage{bbding}         % for \PlusCenterOpen symbol

% TOC customization
% - tocloft for avoiding overlap of chapter number
% - etoc for local table of contents
\usepackage{tocloft}% load it before etoc
\setlength{\cftchapnumwidth}{2em}
\setlength{\cftsecindent}{2em}
\setlength{\cftsecnumwidth}{3.5em}
\usepackage{etoc}
\renewcommand\sphinxtableofcontentshook{}% prevent \sphinxtableofcontentshook
                                         % overriding tocloft

\let\originalsphinxtableofcontents=\sphinxtableofcontents
\renewcommand{\sphinxtableofcontents}{%
   \originalsphinxtableofcontents
   \etocsettocstyle{}{}% tell local toc to not do do anything special before and
                       % after local table of contents
                       % This would be the place to do things such as
                       % adding horizontal lines
}

% \usepackage{xcolor} % (this is done by Sphinx anyhow)

\usepackage{mdframed}
\usepackage{ifthen}

% okvir environment
% TODO: reimplement 'okvir' environment without '\usepackage{fancybox}'
\newenvironment{okvir}[1]{
    \def\okvirWhat{#1}
    \textbf{#1}:
}{
}

% TOC at each chapter
\let\originalchapter=\chapter
\renewcommand*{\chapter}{%
    \secdef{\Chap}{\ChapS}%
}

% start chapter on 'odd' page
\newcommand\ChapS[1]{\originalchapter*{#1}%
                     \addcontentsline{toc}{chapter}{#1}%
                     \localtableofcontents
                     \cleardoublepage
}
\newcommand\Chap[2][]{\originalchapter[#1]{#2}%
                      \localtableofcontents
                      \cleardoublepage
                     }

% start section on new page
\let\stdsection\section
\renewcommand\section{\clearpage\stdsection}

% change subsection style
\let\originalsubsection\subsection
\renewcommand{\subsection}[1]{\originalsubsection*{\PlusCenterOpen\ #1}}

% change subsubsection style
\let\originalsubsubsection\subsubsection
\renewcommand{\subsubsection}[1]{\originalsubsubsection*{\OrnamentDiamondSolid\ #1}}

% change paragraph style
\let\originalparagraph\paragraph
\renewcommand{\paragraph}[1]{\originalparagraph*{\FourStarOpen\ #1}}

% redefine notice environment
\newlength{\noticelength}
\definecolor{bgcolor}{rgb}{0.96, 0.96, 0.89}

% redefine procedure environment
\definecolor{level1c}{RGB}{32,67,92}

% second level itemize style
\renewcommand{\labelitemi}{$\bullet$}
\renewcommand{\labelitemii}{$\circ$}

% don't break lines
\hyphenpenalty 10000
\exhyphenpenalty 10000

\renewenvironment{sphinxadmonition}[2]{
    \vspace{2mm}
    \def\noticeType{#1}
    \def\noticeWhat{#2}

    \setlength{\fboxrule}{1pt}
    \setlength{\fboxsep}{6pt}
    \setlength{\noticelength}{\linewidth}
    \addtolength{\noticelength}{-2\fboxsep}
    \addtolength{\noticelength}{-2\fboxrule}

    \newmdenv[
        backgroundcolor=bgcolor,
        linewidth=1.5pt,
    ]{myframe}

    % warning
    \ifthenelse{\equal{\noticeType}{warning}}
        {\begin{myframe}[linecolor=red]}
    % note
    {\ifthenelse{\equal{\noticeType}{note}}
        {\begin{myframe}[linecolor=blue]}
    % example
    {\ifthenelse{\equal{\noticeType}{admonition}}
        {\begin{myframe}[linecolor=orange]}
    % other
    {\begin{myframe}[linecolor=black]}}}

    \minipage{\noticelength}

    \textbf{#2}
}{
    \endminipage
    \end{myframe}
    \vspace{2mm}
}



\title{test}
\date{Jan 01, 1980}
\release{1.0}
\author{author}
\newcommand{\sphinxlogo}{\vbox{}}
\renewcommand{\releasename}{Release}
\makeindex
\begin{document}

\pagestyle{empty}
\sphinxmaketitle
\pagestyle{plain}
\sphinxtableofcontents
\pagestyle{normal}
\phantomsection\label{\detokenize{index::doc}}


\sphinxstepscope


\chapter{Test01}
\label{\detokenize{ch01/content:test01}}\label{\detokenize{ch01/content::doc}}
\sphinxAtStartPar
Some text…

\sphinxstepscope


\chapter{Test02}
\label{\detokenize{ch02/content:test02}}\label{\detokenize{ch02/content::doc}}
\sphinxAtStartPar
Some text…

\sphinxstepscope


\chapter{Test03}
\label{\detokenize{ch03/content:test03}}\label{\detokenize{ch03/content::doc}}
\sphinxAtStartPar
Some text…

\sphinxstepscope


\chapter{Test04}
\label{\detokenize{ch04/content:test04}}\label{\detokenize{ch04/content::doc}}
\sphinxAtStartPar
Some text…

\sphinxstepscope


\chapter{Test05}
\label{\detokenize{ch05/content:test05}}\label{\detokenize{ch05/content::doc}}
\sphinxAtStartPar
Some text…

\sphinxstepscope


\chapter{Test06}
\label{\detokenize{ch06/content:test06}}\label{\detokenize{ch06/content::doc}}
\sphinxAtStartPar
Some text…

\sphinxstepscope


\chapter{Test07}
\label{\detokenize{ch07/content:test07}}\label{\detokenize{ch07/content::doc}}
\sphinxAtStartPar
Some text…

\sphinxstepscope


\chapter{Test08}
\label{\detokenize{ch08/content:test08}}\label{\detokenize{ch08/content::doc}}
\sphinxAtStartPar
Some text…

\sphinxstepscope


\chapter{Test09}
\label{\detokenize{ch09/content:test09}}\label{\detokenize{ch09/content::doc}}
\sphinxAtStartPar
Some text…

\sphinxstepscope


\chapter{Test10}
\label{\detokenize{ch10/content:test10}}\label{\detokenize{ch10/content::doc}}
\sphinxAtStartPar
Some text…

\sphinxstepscope


\section{Test01}
\label{\detokenize{ch10/subcontent:test01}}\label{\detokenize{ch10/subcontent::doc}}
\sphinxAtStartPar
test


\section{Test02}
\label{\detokenize{ch10/subcontent:test02}}
\sphinxAtStartPar
test


\section{Test03}
\label{\detokenize{ch10/subcontent:test03}}
\sphinxAtStartPar
test


\section{Test04}
\label{\detokenize{ch10/subcontent:test04}}
\sphinxAtStartPar
test


\section{Test05}
\label{\detokenize{ch10/subcontent:test05}}
\sphinxAtStartPar
test


\section{Test06}
\label{\detokenize{ch10/subcontent:test06}}
\sphinxAtStartPar
test


\section{Test07}
\label{\detokenize{ch10/subcontent:test07}}
\sphinxAtStartPar
test


\section{Test08}
\label{\detokenize{ch10/subcontent:test08}}
\sphinxAtStartPar
test


\section{Test09}
\label{\detokenize{ch10/subcontent:test09}}
\sphinxAtStartPar
test


\section{Test10}
\label{\detokenize{ch10/subcontent:test10}}
\sphinxAtStartPar
test


\chapter{Indices and tables}
\label{\detokenize{index:indices-and-tables}}\begin{itemize}
\item {} 
\sphinxAtStartPar
\DUrole{xref,std,std-ref}{genindex}

\item {} 
\sphinxAtStartPar
\DUrole{xref,std,std-ref}{modindex}

\item {} 
\sphinxAtStartPar
\DUrole{xref,std,std-ref}{search}

\end{itemize}



\renewcommand{\indexname}{Index}
\printindex
\end{document}